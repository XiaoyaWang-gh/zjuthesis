\chapter{建设网页体验平台的步骤}

\section{准备接口层}
将智能单测生成方法封装成一个可被网页调用的接口,如 REST API 或 GraphQL。可以使用 Flask、FastAPI(Python)或 Express.js(Node.js)等框架实现后端服务,提供接收代码文件或文本的接口,并返回生成的单测代码。

\section{选择 Web 框架并设计前端}
选择合适的前端框架(如 React、Vue.js 或 Angular)构建网页,设计用户界面,让用户可以在网页上上传代码文件、输入代码,或直接调用单测生成方法,并展示生成的单测代码。

\section{连接前后端}
在前端使用 \texttt{fetch} 或 \texttt{axios} 等方式,将用户输入的数据发送至后端接口,接收并显示生成的单测代码。

\section{部署服务器}
将前后端应用部署至服务器,确保服务可外网访问。可以选择以下部署方案:
\begin{itemize}
    \item 云平台:AWS、Google Cloud、Azure 等提供后端服务和静态网页托管。
    \item 全栈平台:Heroku、Vercel 或 Render 等方便托管前端和后端。
    \item Docker 容器化:使用 Docker 打包应用,便于在服务器上运行和管理。
\end{itemize}

\section{域名和 HTTPS}
申请域名并配置 HTTPS 证书,使用户可以通过安全连接访问网页体验。

\section{用户体验优化}
\begin{itemize}
    \item 输入输出优化:增加示例代码,帮助用户理解如何使用。
    \item 错误处理:处理常见错误,如输入错误、生成失败等。
    \item 响应优化:对较大文件或复杂代码,加入进度条等响应式设计。
\end{itemize}

\section{数据存储与分析(可选)}
如需跟踪使用情况,可将用户上传数据存储至数据库中,并分析用户生成的单测情况,以便进一步优化单测方法。

\chapter{总结}
建设网页体验平台的大致流程为:封装接口 $\rightarrow$ 构建前端 $\rightarrow$ 前后端连接 $\rightarrow$ 部署上线 $\rightarrow$ 优化用户体验。通过此方式,可以为他人提供体验智能单测生成方法的平台。
