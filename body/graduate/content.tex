% \addbibresource{body/ref.bib}



\chapter{引言}
    \section{研究背景}
    \section{研究难点}
    \section{研究现状}
        \subsection{相关测评会议}
        \subsection{相关工作}
    \section{本文的主要工作}
    \section{本文的组织结构}
    \section{本章小结}
    遍地英雄下夕烟~
    % \cite{ni2024casmodatest}

\chapter{相关工作}
    \section{传统软件测试方法}
    \section{基于大模型的软件测试}
    \section{传统方法和大模型结合的软件测试}
    \section{本章小结}
\chapter{基于RAG和LLM的级联单测生成方法}
    \section{方法介绍}
    \section{方法实现}
    \section{本章小结}
\documentclass{book}
\usepackage{ctex} % 添加中文支持
\begin{document}
\chapter{基于智能体的覆盖率引导的单测生成方法}
\section{方法介绍}
\section{方法实现}
\subsection{和上一部分的衔接}
上一章提出的算法CasModaTest产生的最终结果(执行成功的测试用例),将被用作本章提出的算法CovDrivenUTAgent的种子测试用例集合。
本章的想法借鉴了基于遗传算法的


\subsection{整体算法}
\documentclass{article}
\usepackage{amsmath}
\usepackage{algorithm}
\usepackage{algorithmic}

\begin{document}
\begin{algorithm}
    \caption{Test Case Generation}
    \begin{algorithmic}[1]
        \STATE \textbf{Input:} FUT (focal method under test), STC (seed test case) [might be None],
        maxGenCnt (maximum generation count), maxStallLen (maximum stall Length)
        \STATE \textbf{Output:} An archive of test cases
        \STATE genCnt $\gets$ 0
        \STATE stallLen $\gets$ 0
        \IF{STC is None}
        \WHILE{genCnt $\leq$ maxGenCnt}
        \STATE STC $\gets$ LLM(FUT + Strategy\_Two)
        \STATE genCnt $\gets$ genCnt + 1
        \IF{STC is not passed}
        \STATE STC, X $\gets$ feedbackToLLM(NTC)
        \STATE genCnt $\gets$ genCnt + X
        \ELSE
        \STATE \textbf{break}
        \ENDIF
        \IF{STC is passed}
        \STATE \textbf{break}
        \ENDIF
        \ENDWHILE
        \ENDIF
        \IF{STC is None}
        \RETURN [] \COMMENT{The empty tcArchive}
        \ENDIF
        coverageMax $\gets$ calCoverage(STC)
        tcArchive $\gets$ [STC]
        coverageBefore $\gets$ calCoverage(tcArchive)
        \WHILE{genCnt $\leq$ maxGenCnt}
        \IF{stallLen < maxStallLen}
        \STATE NTC $\gets$ LLM(FUT + STC + Strategy\_One)
        \ELSE
        \STATE NTC $\gets$ LLM(FUT + Strategy\_Two)
        \ENDIF
        \IF{NTC is not passed}
        \STATE NTC, X $\gets$ feedbackToLLM(NTC)
        \ENDIF
        \IF{NTC is not passed}
        \STATE genCnt $\gets$ genCnt + X
        \STATE \textbf{continue}
        \ENDIF
        \IF{calCoverage(NTC) > coverageMax}
        \STATE STC $\gets$ NTC
        \STATE coverageMax $\gets$ calCoverage(NTC)
        \ENDIF
        coverageNow $\gets$ calCoverage(tcArchive + NTC)
        \IF{coverageNow == coverageBefore} % 使用双等号
        \STATE stallLen $\gets$ stallLen + 1
        \ELSE
        \STATE tcArchive $\gets$ tcArchive + NTC
        \STATE coverageBefore $\gets$ coverageNow
        \STATE stallLen $\gets$ 0
        \ENDIF
        \STATE genCnt $\gets$ genCnt + 1
        \ENDWHILE
        \RETURN tcArchive
    \end{algorithmic}
\end{algorithm}
\end{document}



\subsection{生成式fuzz}
举例说明生成式fuzz
\subsection{变异式fuzz}
举例说明变异式fuzz
\section{本章小结}
\end{document}
\chapter{实验}
    \section{实验数据}
    \section{基线方法}
    \section{实验设计}
    \section{实验结果}
    \section{结果分析}
    \section{本章小结}
\chapter{建设网页体验平台的步骤}

\section{准备接口层}
将智能单测生成方法封装成一个可被网页调用的接口,如 REST API 或 GraphQL。可以使用 Flask、FastAPI(Python)或 Express.js(Node.js)等框架实现后端服务,提供接收代码文件或文本的接口,并返回生成的单测代码。

\section{选择 Web 框架并设计前端}
选择合适的前端框架(如 React、Vue.js 或 Angular)构建网页,设计用户界面,让用户可以在网页上上传代码文件、输入代码,或直接调用单测生成方法,并展示生成的单测代码。

\section{连接前后端}
在前端使用 \texttt{fetch} 或 \texttt{axios} 等方式,将用户输入的数据发送至后端接口,接收并显示生成的单测代码。

\section{部署服务器}
将前后端应用部署至服务器,确保服务可外网访问。可以选择以下部署方案:
\begin{itemize}
    \item 云平台:AWS、Google Cloud、Azure 等提供后端服务和静态网页托管。
    \item 全栈平台:Heroku、Vercel 或 Render 等方便托管前端和后端。
    \item Docker 容器化:使用 Docker 打包应用,便于在服务器上运行和管理。
\end{itemize}

\section{域名和 HTTPS}
申请域名并配置 HTTPS 证书,使用户可以通过安全连接访问网页体验。

\section{用户体验优化}
\begin{itemize}
    \item 输入输出优化:增加示例代码,帮助用户理解如何使用。
    \item 错误处理:处理常见错误,如输入错误、生成失败等。
    \item 响应优化:对较大文件或复杂代码,加入进度条等响应式设计。
\end{itemize}

\section{数据存储与分析(可选)}
如需跟踪使用情况,可将用户上传数据存储至数据库中,并分析用户生成的单测情况,以便进一步优化单测方法。

\chapter{总结}
建设网页体验平台的大致流程为:封装接口 $\rightarrow$ 构建前端 $\rightarrow$ 前后端连接 $\rightarrow$ 部署上线 $\rightarrow$ 优化用户体验。通过此方式,可以为他人提供体验智能单测生成方法的平台。

\chapter{总结与展望}
    \section{工作总结}
    \section{未来展望}



% \chapter{关于本模板}
% 2024-10-27-test


% 本模板根据浙江大学研究生院编写的《浙江大学研究生学位论文编写规则》~\cite{zjugradthesisrules},
% 在原有的 zjuthesis 模板~\cite{zjutheasis}基础上开发而来。

% 本模板的本科生版本\cite{zjuthesisrules}得到了浙江大学本科生院老师的支持与审核,
% 已经在本科生院网上公示。
% 但当前的研究生版本并未经过研究生院老师的审核,
% 同学们使用时要注意对照模板与要求,
% 切不可盲目使用。

% 作者本人并未编写过浙江大学研究生毕业论文,
% 所以不清楚具体要求。
% 如果有热心同学愿意帮忙,
% 可以替我联系相关老师,我会配合审核并修改代码。

% \section{Overleaf 使用注意事项}

% 如果你在Overleaf上编译本模板,请注意如下事项:

% \begin{itemize}
%     \item 删除根目录的 ``.latexmkrc'' 文件,否则编译失败且不报任何错误
%     \item 字体有版权所以本模板不能附带字体,请务必手动上传字体文件,并在各个专业模板下手动指定字体。
%         具体方法参照 GitHub 主页的说明。
%     \item 当前的Overleaf默认使用TexLive 2017进行编译,但一些伪粗体复制乱码的问题需要TexLive 2019版本来解决。
%         所以各位同学可以在Overleaf上编写论文时务必切换到TexLive 2019或更新版本来编译,以免产生查重相关问题。
%         具体说明参照 GitHub 主页。
% \end{itemize}


% \section{节标题}

% 我们可以用includegraphics来插入现有的jpg等格式的图片,
% 如\autoref{fig:zju-logo}所示。

% \begin{figure}[htbp]
%     \centering
%     \includegraphics[width=.3\linewidth]{logo/zju}
%     \caption{\label{fig:zju-logo}浙江大学LOGO}
% \end{figure}


% \subsection{小节标题}


% \par 如\autoref{tab:sample}所示,这是一张自动调节列宽的表格。

% \begin{table}[htbp]
%     \caption{\label{tab:sample}自动调节列宽的表格}
%     \begin{tabularx}{\linewidth}{c|X<{\centering}}
%         \hline
%         第一列 & 第二列 \\ \hline
%         xxx & xxx \\ \hline
%         xxx & xxx \\ \hline
%         xxx & xxx \\ \hline
%     \end{tabularx}
% \end{table}


% \par 如\autoref{equ:sample},这是一个公式

% \begin{equation}
%     \label{equ:sample}
%     A=\overbrace{(a+b+c)+\underbrace{i(d+e+f)}_{\text{虚数}}}^{\text{复数}}
% \end{equation}




% \chapter{另一章}


% \begin{figure}[htbp]
%     \centering
%     \includegraphics[width=.3\linewidth]{example-image-a}
%     \caption{\label{fig:fig-placeholder}图片占位符}
% \end{figure}

% \chapter{再一章}

% \par 如\autoref{alg:sample},这是一个算法

% \begin{algorithm}[H]
%     \begin{algorithmic} % enter the algorithmic environment
%         \REQUIRE $n \geq 0 \vee x \neq 0$
%         \ENSURE $y = x^n$
%         \STATE $y \Leftarrow 1$
%         \IF{$n < 0$}
%             \STATE $X \Leftarrow 1 / x$
%             \STATE $N \Leftarrow -n$
%         \ELSE
%             \STATE $X \Leftarrow x$
%             \STATE $N \Leftarrow n$
%         \ENDIF
%         \WHILE{$N \neq 0$}
%             \IF{$N$ is even}
%                 \STATE $X \Leftarrow X \times X$
%                 \STATE $N \Leftarrow N / 2$
%             \ELSE[$N$ is odd]
%                 \STATE $y \Leftarrow y \times X$
%                 \STATE $N \Leftarrow N - 1$
%             \ENDIF
%         \ENDWHILE
%     \end{algorithmic}
%     \caption{\label{alg:sample}算法样例}
% \end{algorithm}